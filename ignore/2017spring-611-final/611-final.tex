\documentclass[11pt]{article}
\usepackage[left=0.5in,right=0.5in,top=1in, bottom=.5in]{geometry}
\usepackage{do-template}
\usepackage{palatino}

\makeatletter
\renewcommand\@maketitle{%
	\noindent\begin{minipage}[t]{\linewidth}
				%\vskip 1em
				\let\footnote\thanks 
				{\Large \bf \@title  }
		\end{minipage}\\
		\noindent\begin{minipage}[t]{0.3\linewidth}
				{\large\@author \par}
				{\@date \par }
		\vskip 1em \par
	\end{minipage}
	\vskip 1em
} 
\usepackage{fancyhdr}
\pagestyle{fancy}
\rhead{\thepage}
\cfoot{}
\lhead{Ling 611 $\cdot$ Spring 2017 $\cdot$ Frazier \& Roeper}
\makeatother
\title{The effect of prosodic cues on semantic prediction $\cdot$ Ling 611 final project}
\author{Deniz \"Ozy\i ld\i z}
%\setlength{\parindent}{0pt}
\renewcommand{\focmark}[2]{$\emph{\text{#2}}_\text{#1}$}
\begin{document}
\renewenvironment{flushleft}{\raggedright}{}
\maketitle
\thispagestyle{empty}

\section{Introduction} 
In Turkish, factive attitude reports are realized with a pitch contour that is distinct from that of non-factive attitude reports. 

The difference is in the position of focus (matrix vs. embedded), which is realized by an H- \emph{ip} boundary before the focused word, followed by an H*L contour aligned with its stressed syllable. 
\ex. \gl{Attitude report}\label{ex:fa} 
\gll Tunc [Dilara'nin Ankara'da ol-dug-un-u] bil-iyor.\\
Tunc Dilara\gl{.gen} Ankara\gl{.loc} be\gl{-nonfut.nmz-3s-acc} know\gl{-pres.3s}\\
\glt Tunc \{knows, believes\} that Dilara is in Ankara.\exsep
\a. \gl{Matrix focus}:\\
Tunc Dilara'nin Ankara'da oldugunu)$^{H-}$ (\focmark{F}{bili$^H^*$yor}) \hfill \ldots\ \#ama Ankara'da degil.\\
Tunc knows that Dilara is in Ankara \hfill \ldots\ \#but she's not.\label{ex:f}\exsep
\b. \gl{Embedded focus}\\
Tunc Dilara'nin)$^{H-}$ (\focmark{F}{An$^H^*$kara}'da oldugunu biliyor) \hfill\ldots\ \checkmark ama Ankara'da degil.\\
Tunc believes that Dilara is in Ankara \hfill\ldots\ \checkmark but she's not.\label{ex:nf}\\
\z.

Usually, if a verb `is' factive, both options in \ref{ex:fa} are available---with the associated interpretive difference. If a verb is non-factive (like `believe'), the early focus contour is preferred and the late focus contour is marked.
\ex. \gl{Attitude report}\label{ex:fa2} 
\gll Tunc [Dilara'nin Ankara'da ol-dug-un-u] san-iyor.\\
Tunc Dilara\gl{.gen} Ankara\gl{.loc} be\gl{-nonfut.nmz-3s-acc} believe\gl{-pres.3s}\\
\glt Tunc believes that Dilara is in Ankara.\exsep
\a. \gl{Matrix focus}:\\
\#Tunc Dilara'nin Ankara'da oldugunu)$^{H-}$ (\focmark{F}{sani$^H^*$yor}) \hfill \ldots\ \checkmark ama Ankara'da degil.\footnote{To the extent/in contexts where this sentence is acceptable, the continuation by denial is acceptable.}\\
Tunc believes that Dilara is in Ankara \hfill \ldots\ \checkmark but she's not.\label{ex:f2}\\
(Contrastive focus on `believe'?)\exsep
\b. \gl{Embedded focus}\\
Tunc Dilara'nin)$^{H-}$ (\focmark{F}{An$^H^*$kara}'da oldugunu saniyor) \hfill\ldots\ \checkmark ama Ankara'da degil.\\
Tunc believes that Dilara is in Ankara \hfill\ldots\ \checkmark but she's not.\label{ex:nf2}\\
\z.


Given this pattern, we can ask whether people use prosodic features to predict the semantic properties of upcoming verbs.
\ex. \gl{Hypothesis}
\a. Late focus favors factive verb predictions.
\b. Early focus does not favor factive verb predictions.
\a. Non-factives predicted more often, \gl{or}
\b. No difference in rate of prediction.
\z.

The early focus contour is equally compatible with factive verbs and non-factives. So there might be no preference in this condition. However, non-factives are \emph{only} good with the early focus contour, so people might prefer non-factive predictions in the early focus condition.

It would be interesting to include very early cues as well.

Bigger picture: If there is an effect, we must ask \emph{how} prosodic cues are integrated to make semantic/pragmatic predictions. I'm not perfectly clear on this at this stage.  

\section{Materials \& Methods}
\begin{itemize}
	\item Incomplete attitude reports, with the attitude verb left out, presented auditorily: E.g.,
		\exg. Tunc Dilara'nin Ankara'da oldugunu\\
		Tunc Dilara in~Ankara be\\
		\glt Tunc \gap\ that Dilara is in Ankara. 

	\item Two conditions: Early focus, Late focus.
		\ex. \a. Early focus:\\
		Tunc Dilara'nin) (Ankara'da oldugunu
		\b. Late focus:\\
		Tunc Dilara'nin Ankara'da oldugunu)
		
	\item Forced choice completion task: Factive verb or non factive verb.
	\item Predictions: 
		\begin{itemize}
			\item In the late focus condition, \% factive completions $>$ \% non-factive completions.
			\item In the early focus condition, \% factive $\approx$ \% non-factive.
		\end{itemize} 
	\item 12 experimental items each recorded in EF and LF.
	\item Two lists:
		\begin{itemize}
			\item List 1: Sentence 1 EF, Sentence 2 LF\ldots
			\item List 2: Sentence 1 LF, Sentence 2 EF\ldots
		\end{itemize}
	\item 6 of the 12 items are recorded by author, 6 come from a previous elicitation study and other speakers (3 from one speaker and 3 from another).
	\item Half of the items have know/believe as possible completions, the other half remember/think.
	\item 12 fillers, 2 from author, 5 each from other two speakers. (Participant gets 8 sentences per speaker in total.) Among the fillers:
		\begin{itemize}
			\item 6 root transitives.
			\item 4 attitude reports with intonation patterns unlike experimental items. 
			\item 2 neg raising attitude reports, with matrix negation.
			\item Forced choice in the fillers: imagine/give importance to; see/hear. (There are an equal number of pairs of verbs. 6 san/bil, 6 hatirla/dusun, 6 hayal et/onemse, 6 gor/duy)
		\end{itemize}
	\item Finale: two comprehension questions. Two ARs recorded in EF and LF conditions. Question: Is the embedded proposition true or false?
\end{itemize}

\end{document} 
